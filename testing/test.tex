\section{FTMS}

\subsection{Fourier Transfrom Microwave Spectroscopy Measurement and
Control Suite}

\subsection{Software Specifics}

\textbf{Contributors}: Johannes Rebling

\textbf{Tags:} FTMS, LabVIEW, Fourier Transfrom, Microwave Spectroscopy

\textbf{Requirements Software:} LabVIEW 2011 or newer

\textbf{Requirements Hardware:} OSC, Trigger, Signal Generator, AWG,
GPIB interface,\ldots{}

\textbf{License:} Creative Commons Attribution-NonCommercial-ShareAlike
3.0 Unported License

\textbf{License URL:} http://creativecommons.org/licenses/by-nc-sa/3.0/

This documents describes all software tools that are used to control and
measure microwave spectra using a fourier transform microwave setup as
described below.

\ctable[pos = H, center, botcap]{l}
{% notes
}
{% rows
\FL
\parbox[t]{1.00\columnwidth}{\raggedright
! \# Installation \#\#\#LabVIEW Driver and Libraries To use this
software suite just install the provided LBB file or download the
following device drivers: * OSC driver link * Trigger driver link * PSG
driver link
}
\LL
}

! \# Notes to LabVIEW Software - the timing info stored in the TRIG is
read and display during the startup of \emph{``config\_TRIG.vi''} - when
the timing data is written to the trigger (via the ``write'' button), it
is checked for gross violation of the timing info (see valid timing
data), corrected and then send to the trigger. It is then immediately
read from the trigger and displayed to ensure a correct write process -
the data displayed in \emph{``delay data in delay generator''} is
displayed in seconds and if necessary with a corresponding suffix
(m-milli,u-micro,\ldots{}) - the \emph{``display resolution''} sets the
number of points displayed in the graph and has NO INFLUENCE on the
resolution of the trigger itself

\begin{center}\rule{3in}{0.4pt}\end{center}

\section{Config Timing}

This paragraph will cover how all the instrument in the setup are
triggered by the two Delay generators TRIG1 \& TRIG2. It is strongly
recommended to use the timing diagram together with this documentation
to ensure valid timing values.

The timing is configured using \emph{{[}config\_TRIG.vi{]}} or in the
\emph{Chirp FTMS} software main window by clicking \emph{``Config
Timing''}.

\subsection{Usage}

!\ldots{} initial values loaded from config file specified in the
\emph{Chirp FTMS} software \#\#\# time unit - the data that is send to
and read from the trigger in seconds - internal data handling is also
done in seconds for all sub VIs - the displayed time unit can be chosen
freely - the entered time data (start time, duration, burst period,
molecular delay) is converted to seconds internally and send to the
triggers - all data stored in global variables and the config files is
also in second

!\#\#\# timing diagram The timing diagram does not show the actual TTL
signal for all signals. For signals with negative polarity it shows the
inverse of the signal, that is it shows when the signal is ``active''.

\subsection{Name Conventions \& Dependencies}

\textbf{N - averages} - number of times that the time domain signal
obtained in one meas cycle is averaged to generate the final spectra
\textbf{n - burst count} - number of times that the time domain signal
is measured and average DURING both the background cycle and the
molecular cycle

\textbf{experiment time} duration = number of measurements * meas time -
time for a complete experiment - might include several measurements
\textbf{meas time} duration = N * meas cycle - time for measurement of
one spectrum (approx. 2GHz freq. range) - includes several \emph{meas
cycles} \textbf{meas cycle} duration = background cycle + molec cycle
duration = 2 * (molecular delay + n * burst period) - includes both a
\emph{background cycle} with no molecular pulse and a \emph{molecular
cycle} with a molecular pulse - the averaged background signal is
subtracted from the averaged molecular signal and gives the time domain
signal for one meas cycle - measures one spectra using one molecular
pulse and is averaged over n \emph{bursts} (applied to that one
molecular pulse) and corrected by a background signal measured during
the background cycle \textbf{molecular cycle} duration = molecular delay
+ n * burst period - starts with a nozzle pulse that releases the sample
into the chamber - after the \emph{molecular delay}, n \emph{bursts} are
measured with equal lengths of \emph{burst period} \textbf{background
cycle} duration = molecular delay + n * burst period - exactly equals
the molecular cycle, only there is NO a nozzle pulse, hence there is no
sample in the chamber - after the \emph{molecular delay}, n
\emph{bursts} are measured with equal lengths of \emph{burst period}
\textbf{burst period} - sets the time between \emph{delay cycles} during
a burst - one burst is the smallest unit that actually yields a time
domain signal, with length \emph{burst time} - the burst period is at
least as long as one delay cycle, but might be setup to be longer - the
burst period is determined by the \emph{delay cycle} and is a
experimental specific timing value itself (see below) \textbf{delay
cycle} - consists of the following steps: - activate AMP - trigger AWG
signal - AWG signal is amplified and propagates into the probe chamber -
deactivate AMP - activate SWITCH - trigger OSC - time domain signal is
measured - deactivate SWITCH - the signals within the \emph{delay cycle}
are defined by a individual start time (relative to the burst start) and
an individual duration relative to the corresponding start times - the
length of the delay cycle is determined by the experimental specific
timing values that it consists of (see below)

\subsection{Sample and Experimental Specific Timing Values \&
Restrictions}

The timing values of a single burst depend on the sample that is used as
well as other experimental conditions (pressure, temperatures,\ldots{})
and should be determined experimentally before starting a full spectral
measurement. The burst timing values within are controlled by TRIG2 and
can be configured by the \emph{Config Timing} VI. Please refer to the
timing diagram for a graphical display of the complex timing process.
The experimental specific timing values include: - molecular delay (A1
to B1 \& C1 to D1) - time between the nozzle trigger and the start of
n-burst periods - this time is needed to allow the nozzle driver to
actually open the nozzle, let the sample flush into the chamber and
expand there so it can then be radiated by the MW chirp - burst period -
the burst period may range from 100 ns to 2000 -- 10 ns in 10 ns steps
(hardware limits) - while the minimum burst period is restricted by the
\emph{delay cycle}, its maximum is limited by the experimental settings
- you want to measure as many burst in one meas cycle as possible, since
this will significantly improve your signal to noise ratio and will
allow for faster measurements with/or a high resolution - hence you want
to make the burst period as short as possible, however, it must be -
delay cycle - the length of the delay cycle equals the end of the
longest delay generated by TRIG2 (e.g.~the end of the SWITCH pulse, F2)
plus 25ns!

\subsection{Delay Cycle Durations and Start Times}

AMP start (A2) and duration (A2 to B2)

\begin{verbatim}
* activates the power amplifier for a specific duration
* hast to be activated first, hence A2 is usually equal to 0
* duration depends on the chirped pulse length created by the AWG, which depends on the experiment
\end{verbatim}

AWG start (C2) and duration (C2 to D2)

\begin{itemize}
\item
  triggers the chirped pulse generated by the AWG
\item
  AWG is activated via a rising edge, hence the duration is not that
  important
\item
  the AMP must be switched on before the AWG, otherwise the chirp signal
  will not be send to the chamber since the power amp is still switch
  off
\item
  the power amp must be activated for the full duration of the AWG
  signal (check waveform length using the \emph{``ArbExpress
  Application''}, approx. 4us)
\end{itemize}

SWITCH start (E2) and duration (E2 to F2)

\begin{itemize}
\item
  activates the protective switch after the probe chamber
\item
  this switch protects the subsequent high gain MW amplifiers (which
  amplify the frequency signal radiated by the sample molecules), since
  they would be destroyed by the high power MW signal created by the AMP
\item
  the AMP must be deactivated before the SWITCH can be activated
\item
  the switch must be activated long enough to allow the frequency signal
  to be measured by the OSC
\end{itemize}

OSC start (G2) and duration (G2 to H2)

\begin{itemize}
\item
  triggers a measurement of the OSC
\item
  triggered on the rising edge, hence the duration is not that important
\item
  the waveform length in the OSC must be set high enough to measure the
  complete frequency signal
\item
  must be set equal or less than the SWITCH duration, otherwise you
  don't measure anything anyways\ldots{}
\end{itemize}

\textbf{Restrictions summarized:}

\begin{itemize}
\item
  C2(AWG trigger) \textgreater{}= A2(AMP start)
\item
  B2(AMP stop) \textgreater{} Chirp Pulse in AWG
\item
  E2(SWITCH start) \textgreater{} B2(AMP stop)
\item
  OSC waveform length = SWITCH duration
\item
  G2(OSC trigger) \textless{}= B2(SWITCH start)
\item
  burst period \textgreater{}= delay cycle, as short as possible
\end{itemize}
